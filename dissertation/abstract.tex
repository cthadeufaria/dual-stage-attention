%% abstract.tex: abstract in PT and EN  (FEUP regulations)
%% -------------------------------------------------------
\chapter*{Resumo}
Este relatório aborda a concepção, desenvolvimento e avaliação de um sistema automatizado de avaliação da qualidade de vídeo, orientado por algoritmos de \textit{machine learning} e especificamente adaptado para fluxos audiovisuais em ambientes de produção profissional. O estudo propõe uma arquitetura baseada no modelo Dual-Stage Attention (DSA-QoE) para prever a Qualidade de Experiência (QoE) de forma contínua e sem referência, integrando características semânticas de alto nível com indicadores de qualidade de serviço (QoS) de baixo nível. O trabalho é dividido em duas fases complementares: a fase preparatória compreende uma análise crítica da literatura, a avaliação comparativa de modelos como FastVQA, MDVSFA e DSA-QoE, e a seleção de ferramentas e bancos de dados otimizados para a tarefa de VQA; a fase de desenvolvimento visa a implementação da inferência do modelo DSA-QoE em um pipeline de mídia escrito inteiramente em Rust, destacando as vantagens de segurança de memória e concorrência determinística dessa linguagem. Avaliações objetivas e subjetivas estão planejadas para validar o desempenho do sistema em termos de latência, throughput e aderência perceptual. A principal contribuição deste trabalho reside na validação empírica da viabilidade de usar Rust para inferência em tempo real de modelos de visão computacional complexos, ao mesmo tempo em que se avança o estado da arte em avaliação de qualidade audiovisual em cenários de streaming.

\chapter*{Abstract}
This report presents the design, development, and evaluation of an AI-driven video quality assessment system, tailored for professional media workflows. The proposed approach centers on the Dual-Stage Attention (DSA-QoE) model for continuous, no-reference Quality of Experience (QoE) prediction, combining high-level semantic features with low-level Quality of Service (QoS) indicators via hierarchical temporal attention and adaptive fusion. The project unfolds in two complementary phases: the preparatory phase offers a systematic literature review, comparative model analysis—including FastVQA, MDVSFA, and DSA-QoE—and the selection of optimized datasets and tooling; the development phase focuses on deploying the DSA-QoE model’s inference pipeline in Rust, leveraging the language’s memory safety guarantees and deterministic concurrency model. The system will undergo rigorous objective and subjective testing to evaluate its suitability for real-time deployment in streaming applications. The dissertation's core contribution lies in demonstrating the operational feasibility of using Rust for inference with state-of-the-art deep learning models, while simultaneously advancing the field of perceptual quality assessment in live audiovisual distribution scenarios.
