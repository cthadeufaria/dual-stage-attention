%% abstract.tex: abstract in PT and EN  (FEUP regulations)
%% -------------------------------------------------------
\chapter*{Resumo}
Este trabalho desenvolve e avalia um sistema de inferência contínua da Qualidade de Experiência (QoE) para fluxos audiovisuais, combinando redes neurais de atenção com uma implementação em tempo real usando Rust. A pesquisa parte da constatação de que métricas tradicionais de qualidade, como PSNR e SSIM, não capturam adequadamente a percepção humana em cenários de transmissão com degradações dinâmicas, como rebuffering e variações de bitrate. Para abordar essas limitações, é adotado o modelo Dual-Stage Attention (DSA-QoE), que funde características semânticas e indicadores de Qualidade de Serviço (QoS) usando mecanismos de atenção temporal hierárquicos.

A arquitetura proposta foi treinada no conjunto de dados LIVE-NFLX-II e demonstra alta correlação com os escores subjetivos de QoE, tanto em termos de precisão absoluta (RMSE) quanto de ordenação perceptiva (SRCC). O modelo foi integrado a uma pipeline de inferência nativa em Rust, com ingestão de vídeo via GStreamer e execução do modelo TorchScript por meio da biblioteca \texttt{tch-rs}. O sistema atinge latência sub-real em todas as fases do processamento, realizando inferências em menos de 964\,ms por segundo de vídeo, garantindo viabilidade para aplicações de streaming ao vivo.

O trabalho contribui para a pesquisa em avaliação perceptual e engenharia de sistemas, demonstrando a viabilidade de modelos profundos de atenção em ambientes de produção com restrições de desempenho. Os resultados obtidos indicam que é possível unir acurácia perceptual e garantias de segurança e concorrência através da adoção de Rust, estabelecendo uma base para futuras extensões em dispositivos de borda e fluxos com requisitos mais rígidos.

\textbf{Palavras-chave:} Qualidade de Experiência; Rust; Inferência em Tempo Real; Atenção Temporal; Streaming.


\chapter*{Abstract}
This dissertation presents the development and evaluation of a continuous Quality of Experience (QoE) inference system for audiovisual streaming, combining attention-based neural architectures with a real-time Rust-based implementation. The motivation stems from the inadequacy of traditional quality metrics such as PSNR and SSIM to reflect human perception under real-world streaming impairments, including rebuffering and bitrate shifts. To address this, the Dual-Stage Attention (DSA-QoE) model is adopted, integrating semantic content features and dynamic Quality of Service (QoS) indicators through hierarchical temporal attention mechanisms.

Trained on the LIVE-NFLX-II dataset, the model achieves high correlation with subjective QoE labels, demonstrating both perceptual accuracy and generalization across diverse content. The inference pipeline, implemented natively in Rust, employs GStreamer for real-time video ingestion and executes the TorchScript model using the \texttt{tch-rs} crate. End-to-end processing completes in under 964\,ms per 1-second video chunk, meeting the latency constraints of real-time streaming applications.

This work contributes to the fields of perceptual modeling and systems engineering by validating the feasibility of deploying deep attention-based QoE models in production environments. The results establish a pathway for unifying perceptual fidelity with safe and efficient systems programming, paving the way for future adaptations in edge computing and resource-constrained streaming pipelines.

\textbf{Keywords:} Quality of Experience; Rust; Real-Time Inference; Temporal Attention; Streaming.