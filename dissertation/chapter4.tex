\chapter{Conclusions} \label{chap:ch4}

This chapter synthesizes the main insights gained from the preparatory phase and articulates the expected contributions of the dissertation. It also reflects on the anticipated outcomes and potential implications of the planned methodologies.

\section{Understanding the Problem}

Video Quality Assessment (VQA) resides at the nexus of computational efficiency and perceptual fidelity. Traditional full‐reference metrics such as Peak Signal‐to‐Noise Ratio (PSNR) and Structural Similarity Index (SSIM) provide convenient numerical scores but correlate poorly with human judgments, particularly under complex, real‐world distortions ~\cite{huynh2012scope} ~\cite{ wang2004ssim}. These limitations are exacerbated in live streaming scenarios, where reference frames are unavailable and artifacts such as rebuffering stalls and bitrate fluctuations dominate user experience. 

No‐reference VQA methods address this gap by inferring Quality of Experience (QoE) directly from distorted signals, yet conventional approaches relying on handcrafted features struggle to capture the joint effects of spatial impairments and temporal dynamics ~\cite{min2024perceptual}. Data‐driven architectures such as MDVSFA and FastVQA advance perceptual alignment and cross‐dataset generalization, but they remain predominantly evaluated in offline settings and lack explicit modeling of playback‐level impairments.

Streaming environments impose stringent real‐time constraints: inference latency must not exceed the inter‐frame interval, and system architectures must safely handle high‐throughput media pipelines. This creates a dual challenge: (1) designing models that accurately predict human QoE under both spatial distortions and network‐induced artifacts, and (2) integrating these models into production‐grade software stacks with deterministic performance and memory safety.

To address these challenges, this dissertation focuses on the dual‐stage attention QoE model of Jia \textit{et al.} ~\cite{jia2024continuous}, which unifies high‐level semantic features and low‐level QoS indicators through hierarchical temporal attention and adaptive fusion. The core question is whether such an advanced model can be deployed in a Rust‐based real‐time media pipeline, balancing perceptual accuracy with the operational requirements of professional streaming services.  

\section{Preliminary Insights and Findings}

Although the full implementation phase has not yet begun, the preparatory activities conducted thus far have offered key technical and conceptual insights relevant to the feasibility of real-time QoE inference in a professional media context. The architectural analysis in Chapter~\ref{chap:ch2} led to a refinement of the original model selection strategy. While MDVSFA~\cite{li2023unified} and FastVQA~\cite{wu2022fastvqa} were initially considered for their robustness and computational efficiency, the dissertation has since narrowed its focus exclusively to the dual-stage attention model (DSA-QoE) proposed by Jia \textit{et al.}~\cite{jia2024continuous}. This decision reflects the model's superior alignment with perceptual quality indicators in streaming contexts and its unified design for both continuous and overall QoE prediction.

From a systems perspective, the choice of Rust as the deployment language has proven strategically justified. Initial tests integrating TorchScript ~\cite{paszke2019pytorch} inference within a Rust environment via the \texttt{tch-rs} crate confirm that it is feasible to load deep learning models trained in PyTorch ~\cite{paszke2019pytorch} and execute them with deterministic memory and concurrency control. Rust’s ownership model and lack of a garbage collector allow for predictable latency, a critical factor in media pipelines where jitter and frame drops can degrade perceived quality. These properties position Rust as a viable alternative to Python or C++ in production-grade AI systems, particularly for scenarios requiring concurrent stream handling and safety guarantees.

Conceptually, the preparatory review of state-of-the-art models and performance trade-offs has deepened the understanding of QoE assessment challenges. Notably, the dual-stage attention model’s capacity to integrate semantic content and time-varying QoS features through short and long-term regression modules offers a principled way to approximate user-perceived degradation under fluctuating streaming conditions. The model’s use of Cross-Feature Attention (CFA) and Self-Adaptive Fusion provides a strong basis for handling perceptual non-linearities, and its codec-agnostic design aligns with the heterogeneity of modern streaming workflows.

These findings validate the dissertation’s strategic direction: to treat QoE inference not merely as a model selection problem, but as an end-to-end system integration challenge. The work to date establishes a foundation for the next phase, in which the dual-stage attention model will be implemented, deployed, and evaluated under realistic production constraints using a Rust-based media pipeline.

\section{Contributions of the Dissertation}

This dissertation contributes to the field of perceptual quality assessment and real-time AI deployment in two complementary domains: algorithmic modeling and systems engineering. From a modeling perspective, it adopts and implements the dual-stage attention architecture proposed by Jia \textit{et al.}~\cite{jia2024continuous}, which represents a state-of-the-art Transformer-based design for predicting both continuous and overall Quality of Experience (QoE). The model combines semantic and motion feature extraction via ResNet-50 and SlowFast backbones with hierarchical temporal modeling through short and long-term regression modules, Cross-Feature Attention (CFA), and a self-adaptive fusion mechanism. This work builds upon that architecture by extending its application to professional-grade media workflows, assessing its inference performance under operational constraints, and adapting it for integration into a production context.

On the systems side, the dissertation provides an empirical validation of Rust as a deployment environment for real-time QoE prediction based on deep learning. While most prior work in the field assumes deployment in Python or C++ ecosystems, this work evaluates whether Rust’s deterministic memory management, zero-cost abstraction, and concurrency model are sufficient to support latency-sensitive inference in high-throughput streaming pipelines. By integrating TorchScript-serialized models with the \texttt{tch-rs} crate and building a full audiovisual inference pipeline in Rust, the project yields a systematic study of throughput, latency, and memory consumption under conditions representative of live broadcast workloads.

Additionally, the dissertation contributes a comprehensive development methodology, covering the complete pipeline from model training and adaptation to deployment and evaluation. The insights generated from this effort will serve as a practical resource for future researchers and engineers aiming to bridge deep learning and systems integration in multimedia contexts.

Finally, by aligning perceptual modeling with the constraints of real-time streaming and embedding it in a systems programming paradigm, this work helps bridge the gap between theoretical research in QoE and practical requirements in industrial media delivery. It will help demonstrate that deep, attention-based models can be deployed safely and efficiently without sacrificing interpretability or responsiveness, thus paving the way for new hybrid pipelines that combine perceptual accuracy with systems reliability.

\section{Anticipated Outcomes and Future Directions}

The anticipated outcomes of this dissertation include the development and validation of a scalable, real-time framework for AI-driven video quality assessment (VQA), implemented in a systems-level language and tailored for professional audiovisual environments. This framework is expected to bridge the gap between state-of-the-art deep learning–based QoE models and the practical demands of live streaming infrastructures, offering a robust solution for latency-constrained, multi-stream scenarios.

A key anticipated contribution is the empirical characterization of performance trade-offs—including latency, throughput, memory consumption, and perceptual accuracy—when deploying Transformer-based models for QoE in real-world conditions. The use of Rust in this context is also expected to yield valuable insights into systems-level design for multimedia AI applications, particularly regarding concurrency, determinism, and memory safety.

Despite these advances, one of the most significant challenges in current VQA research remains the generalization capacity of models across datasets with different distortion types, resolutions, content semantics, and subjective scoring distributions. Cross-dataset performance degradation is widely recognized in the literature, and addressing it is essential for building deployable models that operate reliably in diverse environments. In this regard, recent work such as the Unified Quality Assessment framework proposed by Li \textit{et al.}~\cite{li2023unified} offers promising strategies. Their approach leverages mixed-dataset training to align subjective scales and reduce domain shift, thus improving performance on out-of-distribution test sets as in the case of real world professional media pipelines.

Future directions should thus explore integrating such strategies into the training of the dual-stage attention or other similar model of choice. While the present work focuses on model deployment and feasibility in production environments, subsequent research could investigate how to extend the model’s robustness via unified training schemes, potentially enhancing its utility in both professional and user-generated content (UGC) domains.

Additionally, future work may examine lightweight model variants optimized for edge devices or low-resource settings, thereby extending the applicability of high-fidelity QoE prediction beyond centralized infrastructures.

\section{Summary}

In conclusion, this dissertation aims to address critical challenges in video quality assessment by leveraging state-of-the-art AI algorithms and traditional metrics. The preparatory work has established a solid foundation, and the planned methodologies promise to deliver meaningful contributions to the field. By bridging the gap between computational efficiency and perceptual fidelity, this work aspires to advance the state-of-the-art in VQA technologies.

This chapter has consolidated the core objectives, preliminary insights, and anticipated outcomes of the dissertation. Building upon the architectural synthesis in Chapter~\ref{chap:ch2} and the implementation roadmap in Chapter~\ref{chap:ch3}, this work seeks to operationalize a state-of-the-art QoE prediction model—specifically the dual-stage attention network of Jia \textit{et al.}~\cite{jia2024continuous}—within a real-time, Rust-based media processing environment. Early investigations into model feasibility, systems integration, and performance constraints have confirmed the viability of this approach and highlighted Rust’s value in providing memory-safe, low-latency inference pipelines.

The contributions of this work are multifaceted: they span the technical domain of high-fidelity perceptual modeling and the systems engineering domain of robust AI deployment. By rigorously evaluating the trade-offs between latency, accuracy, and scalability in real-world conditions, the dissertation aims to provide actionable insights for the integration of deep learning–based QoE solutions into professional streaming infrastructures.

Furthermore, this chapter has underscored the ongoing challenge of cross-dataset generalization in VQA research and pointed toward unified training methodologies, such as those proposed by Li \textit{et al.}~\cite{li2023unified}, as promising avenues for future investigation. Looking ahead, extensions to different deployment scenarios (e.g., edge inference) offer fertile ground for expanding the impact and applicability of this work.

In summary, the dissertation positions itself at the intersection of perceptual modeling, systems design, and practical media engineering. Through its empirical focus and methodological rigor, it aims to advance both the academic understanding and industrial deployment of real-time QoE prediction systems.
